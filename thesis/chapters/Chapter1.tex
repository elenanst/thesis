\chapter{\scshape{Εισαγωγή}}
\section{Γενικά}
Όταν ο Arthur Lee Samuel εισήγαγε τον όρο \textit{μηχανική μάθηση}, το 1959, μάλλον δεν ανέμενε την ταχεία εξέλιξή του σε τομέα με τεράστιο επιστημονικό ενδιαφέρον, εμπορική σημασία και καθολική αναγνωρισιμότητα. Μία δραστήρια κοινότητα μαθηματικών, αναλυτών δεδομένων, μηχανικών και προγραμματιστών έχει τροφοδοτήσει τη βιβλιογραφία με πληθώρα αλγορίθμων και τεχνικών μηχανικής μάθησης, την αγορά με εφαρμογές της και την κοινωνία με τη δυνατότητα, ή απειλή, της Τεχνητής Νοημοσύνης.

Τα τελευταία χρόνια έχει κυριαρχήσει η εικόνα της παγίωσης των τεχνικών μηχανικής μάθησης. Η πληθώρα των διαθέσιμων δεδομένων και η αναγνώριση της επιστημονικής και εμπορικής αξίας τους έχει αναδείξει απαιτητικά προβλήματα μηχανικής μάθησης, τάση απέναντι στην οποία η κοινότητα ανταποκρινόταν με τη σχεδίαση νέων τεχνικών και αλγορίθμων. Σήμερα ωστόσο ένα μεγάλο μέρος της βιβλιογραφίας αναλώνεται στην προσπάθεια βελτιστοποίησης προβλημάτων με ιδιαιτερότητες, η επιτυχής επίλυση των οποίων ενδεχομένως επιφέρει εμπορικά, ανθρωπιστικά ή άλλου είδους οφέλη, που όμως δεν μπορούν να επεκταθούν πέρα από την προφανή επίλυση του προβλήματος. Η ουσία της δυσλειτουργικότητας έγκειται στο γεγονός ότι δεν έχει παραχθεί γνώση χρήσιμη για την επιστήμη της μηχανικής μάθησης, καθώς η προσέγγιση που έχει ακολουθηθεί δεν είναι μεταφέρσιμη σε άλλα προβλήματα Προκύπτει λοιπόν το εξής ερώτημα: ήρθε η ώρα να περάσουμε σε ένα \textit{νέο στάδιο μάθησης};

Η μετα-μάθηση έλαβε υπόσταση το 1992, με την εμφάνιση των πρώτων συστημάτων \citep{craw1993,Brazdil1994}, που επιχειρούσαν να αυτοματοποιήσουν στάδια της εξόρυξης δεδομένων, όπως η επιλογή αλγορίθμου μάθησης. Κλειδί στην προσέγγιση της μετα-μάθησης αποτελεί η προσπάθεια συλλογής εμπειρίας από ένα σύστημα με μορφή γνώσης, παραγόμενης από παρελθοντικά πειράματα. Πρόκειται για έναν ευρύ τομέα, που σήμερα αποτελεί εφαλτήριο για την εξέλιξη της μηχανικής μάθησης.


Αν και βήματα προς την αναθεώρηση της συμβατικής εφαρμογής μηχανικής μάθησης γίνονται από το 1995 (Ενότητα \ref{section:autohistory}), η συνειδητοποιημένη κινητοποίηση της κοινότητας προς την αυτοματοποίηση ης μηχανικής μάθησης ξεκίνησε πολύ αργότερα, με τους πρώτους διαγωνισμούς να κάνουν την εμφάνισή τους το 2011~\footnote{http://automl.chalearn.org/}. Εν έτει 2017 η κοινότητα προσπαθεί να ορίσει τη νέα τάση στη μηχανική μάθηση, το \textit{AutoML}.

Ο κλάδος του \textit{AutoML}, πατώντας στην εμπειρία δεκαετιών εφαρμογής μηχανικών μάθησης, χρησιμοποιώντας τεχνικές που πηγάζουν από αυτήν, αλλά και άλλες επιστήμες, προσπαθεί να αντιμετωπίσει τα απαιτητικά προβλήματα, που απασχολούν την τρέχουσα αγορά, με μία νέα προσέγγιση: μαθαίνοντας στις μηχανές να μαθαίνουν, όχι πλέον την επίλυση μεμονωμένων προβλημάτων, αλλά \textit{την ίδια τη διαδικασία της μάθησης}. 


\section{Μεθοδολογία και στόχοι} Μία σύντομη ματιά στη βιβλιογραφία του AutoML αποκαλύπτει την επιτακτικόττα της ανάγκης σχεδιασμού και υλοποίησης εργαλείων μηχανικής μάθησης, τα οποία υποβοηθούν τον αναλυτή δεδομένων αυτοματοποιώντας χρονοβόρες διαδικασίες και επιστρατεύοντας μηχανισμούς εξαγωγής μετα-γνώσης. Η παρούσα εργασία στοχεύει στην αναγνώριση και αντιμετώπιση κενών, καθώς και την εκμετάλλευση δυνατοτήτων στο χώρο του AutoML μέσω της υλοποίησης ενός συστήματος αυτόματης ανάλυσης δεδομένων. Το σύστημα θα αναλαμβάνει τη βελτιστοποίηση προβλημάτων δυαδικής ταξινόμησης έχοντας ως στόχο την προσομοίωση ενός πραγματικού αναλυτή δεδομένων, εξοπλισμένου με εργαλεία αυτοματοποίησης των τεχνικών μηχανικής μάθησης που χρησιμοποιεί.

Πρώτο βήμα στη προσέγγιση του προβλήματος αποτέλεσε ο εντοπισμός των σημείων στη διαδικασία της μηχανικής μάθησης που επιδέχονται και χρήζουν αυτοματοποίησης. Αναγνωριστικά αυτών των σημείων είναι η χρονική και υπολογιστική επιβάρυνση και η αναγνώριση κάποιου μηχανισμού βελτιστοποίησης του προβλήματος μέσω μαθηματικής διατύπωσής του. Χαρακτηριστικό παράδειγμα αυτής της κατηγορίας είναι η επιλογή βέλτιστων υπερ-παραμέτρων κατά τη ρύθμιση ενός μοντέλου μηχανικής μάθησης.

Το σύστημά μας θέτει ιδιαίτερη βαρύτητα στην τεχνική με την οποία γίνεται η βέλτιστη επιλογή υπερ-παραμέτρων ερευνώντας δύο άξονες αυτού του αντικειμένου: διαθέσιμους αλγορίθμους βελτιστοποίησης και τρόπους ενσωμάτωσης μετα-γνώσης στη διαδικασία. Το σύστημά μας υποστηρίζει και πειραματίζεται με συμβατικές (πλεγματικής αναζήτηση) και πρωτοποριακές (bayesian βελτιστοποίηση) τεχνικές βελτιστοποίησης, τις οποίες αξιολογεί μέσω στατιστικών σετ υπόθεσης.

Η σημασία της μετα-γνώσης στη μηχανική μάθηση έχει αναγνωριστεί μέσα στη  γενικότερη προσπάθεια αυτοματοποίησης, καθώς έχει αναδυθεί η ανάγκη και δυνατότητα εκμετάλλευσης της εμπειρίας που δημιουργείται με την επιτυχημένη αντιμετώπιση προβλημάτων. Αναγνωρίζουμε πως η υλοποίηση ενός λογισμικού ανάλυσης δεδομένων αποτελεί σημαντική ευκαιρία εκμετάλλευσης της θεωρίας της μετα-γνώσης για την επίτευξη ενός έμπειρου, εκπαιδευόμενου και επεκτάσιμου προγράμματος. Έτσι, το σύστημά μας ενσωματώνει τη χρήση μετα-χαρακτηριστικών για τη πρόβλεψη υπερ-παραμέτρων, μια τεχνική που μας απαλλάσσει από την ανάγκη βελτιστοποίησής τους. 

Αποτελεί, πλέον, κοινή παραδοχή ότι η επιτυχημένη μηχανική μάθηση προϋποθέτει τη χρήση, ή έστω τον πειραματισμό με ποικιλία τεχνικών. Φαίνεται πως η κοινότητα της μηχανικής μάθησης έχει αρχίσει να θέτει υπό αμφισβήτιση την αρχή της απλότητας του μοντέλου μάθησης, γνωστής ως "ξυράφι του Όκαμ" για να περάσει στη πλευρά του Επίκουρου, σύμφωνα με τον οποίο "ο συνδυασμός σωστών λύσεων σε ένα πρόβλημα, δε μπορεί παρά να λύνει το πρόβλημα τουλάχιστον εξίσου καλά". Η μεταφορά βέβαια αυτής της αρχής στο χώρο της μηχανικής μάθησης απαιτεί ιδιαίτερη προσοχή κατά την αξιολόγηση του μοντέλου, καθώς ενέχει ο κίνδυνος υπερ-προσαρμογής, ένα πρόβλημα που αναλύεται στο Παράρτημα \ref{appendix:Reg}.

Αυτή η διαπίστωση αποτέλεσε βασικό παράγοντα στον καθορισμό της λειτουργικότητας του συστήματός μας, το οποίο υποστηρίζει πληθώρα αλγορίθμων και τεχνικών προ-επεξεργασίας και ανάλυσης δεδομένων, θέτοντας την απαίτηση για τη χρήση συλλογών μοντέλων. Δεδομένης της απαιτητικότητας που δημιουργεί η παρουσία πολλών, ενδεχομένως ποιοτικά αμφισβητήσιμων μοντέλων, ενσωματώσαμε την τεχνική του σχηματισμού συλλογών μοντέλων με προς τα εμπρός επιλογή \citep{Caruana:2004:ESL:1015330.1015432}, μία προσέγγιση που έχει ξαναχρησιμοποιηθεί σε σχετικές εργασίες.

Τέλος, υπάρχει μία προσέγγιση της μηχανικής μάθησης, η οποία δεν μπορεί να βελτιστοποιηθεί, να προκύψει από μετα-γνώση ή την εφαρμογή κάποιου αλγορίθμου μάθησης, αλλά συνιστά απαραίτητο εργαλείο στα χέρια του αναλυτή δεδομένων. Πρόκειται για τη χρήση ευριστικών. Θεωρούμε πως η παράλειψη ενσωμάτωσης ευριστικών θα στερούσε από το σύστημά μας πρακτική γνώση, απαραίτητη για τη λήψη σχεδιαστικών αποφάσεων. Έχουμε επομένως αναζητήσει και συλλέξει ευριστικές από τη βιβλιογραφία, τις οποίες ενσωματώσαμε στο λογισμικό, παραμετροποιώντας σχεδιαστικές αποφάσεις που παίρνει ο αλγόριθμος.

\section{Διάρθρωση Κειμένου} Η εργασία αποτελείται από 6 κεφάλαια, συμπεριλαμβανομένου και του παρόντος βιβλιογραφικού.

Στο Κεφάλαιο 2 θέτουμε το θεωρητικό υπόβαθρο, στο οποίο βασίστηκε το σύστημα μας. Συγκεκριμένα ορίζουμε τη διαδικασία της μηχανική μάθησης και αναλύουμε βασικές τεχνικές της. Στη συνέχεια εισάγουμε τον αναγνώστη στο χώρο του AutoML, παραθέτοντας ιστορικά στοιχεία, γνωρίζοντας το τρέχων state of the art και αναλύοντας δύο εκφάνσεις αυτής της επιστήμης, που αξιολογήσαμε ως κυρίαρχες στη βιβλιογραφία: τη βελτιστοποίηση των υπερ-παραμέτρων μοντέλων μηχανικής μάθησης και τη μετα-μάθηση.  

Στο Κεφάλαιο 3 παρατηρούμε αναλυτικά το σύστημα μας Αρχικά παραθέτουμε τα κίνητρα που οδήγησαν στη σχεδίαση του συστήματος και διαμόρφωσαν τα κύρια χαρακτηριστικά του κατά την εκκίνηση της εργασίας. Ακολουθεί αναλυτική περιγραφή των τεχνικών που επιστρατεύτηκαν για την επίτευξη της επιθυμητής λειτουργικότητας, οι οποίες, εμπνευσμένες από τη πρόσφατη βιβλιογραφία όφειλαν να προσαρμοστούν στο σύστημα και να εξεταστεί η συνεισφορά τους. Τέλος, παραθέτουμε τη βασική αρχιτεκτονική του λογισμικού που σχεδιάσαμε αναλύοντας τα δύο βασικά υποσυστήματά του: το υποσύστημα εκπαίδευσης και το υποσύστημα πειράματος. Πιστεύουμε πως η ανάλυση αυτή θα βοηθήσει στην κατανόηση της ενσωμάτωσης των τεχνικών που έχουμε αναλύσει.

Το Κεφάλαιο 4 επιχειρεί να αξιολογήσει το σύστημα ως ολότητα, καθώς και μεμονωμένες τεχνικές που χρησιμοποιεί, θεωρώντας τες ως υποσυστήματα. Αρχικά περιγράφουμε τη διαδικασία συλλογής των σετ δεδομένων που χρειαστήκαμε για την ανάλυση, καθώς και παραμετροποιήσεις εργαλείων που χρησιμοποιήσαμε, ώστε να είναι εφικτή η αναπαραγωγή των πειραμάτων. Στη συνέχεια αξιολογούμε το υποσύστημα που ασχολείται με τη ρύθμιση των μοντέλων, το υποσύστημα του ensemble και τέλος, το συνολικό σύστημα. Σε κάθε περίπτωση αναφέρουμε σχεδιαστικές επιλογές και προ-απαιτούμενα του πειράματος.

Στο Κεφάλαιο 5 αναφέρουμε δημοσιεύσεις, στις οποίες βασιστήκαμε και περιγράφουμε τη λειτουργία συστημάτων παρόμοιων με το δικό μας.

Στο Κεφάλαιο 6 αναθεωρούμε τη λειτουργία του συστήματός μας εξάγοντας γενικότερα συμπεράσματα από την πειραματική αξιολόγηση και τοποθετούμε το σύστημα ως προς τη συνεισφορά του στη σύγχρονη βιβλιογραφία

Στο Κεφάλαιο 7, αφορμώμενοι από περιορισμούς, προβλήματα και ιδέες που προέκυψαν στη διάρκεια της εργασίας μας, παραθέτουμε μελλοντικές επεκτάσεις-βελτιώσεις του συστήματος.