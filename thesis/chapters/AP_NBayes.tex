\chapter{Naive Bayes}
\label{appendix:NBayes}
\paragraph{Θεώρημα Bayes.} Μία ακόμη πηγή έμπνευσης για το πρόβλημα της ταξινόμησης βρίσκεται στην επιστήμη των πιθανοτήτων. Ο αλγόριθμος Naive Bayes δίνει απάντηση στο ερώτημα:”Δεδομένων των
παραδειγμάτων που έχω, ποια είναι η πιθανότερη υπόθεση;” κάνοντας χρήση του θεωρήματος Bayes, το οποίο στην περίπτωσή μας διατυπώνεται ως εξής: 
$$P(h \mid d)= \frac{P(d \mid h) P(h)}{P(d)}$$
όπου h είναι η υπόθεση και d τα παραδείγματα.

Ως γνωστόν, σε ένα πρόβλημα ταξινόμησης η υπόθεση ισοδυναμεί με την κλάση που αναθέτουμε σε ένα παράδειγμα. Ας δούμε λίγο πιο αναλυτικά τις πιθανότητες, με τις οποίες ασχολούμαστε:
\begin{itemize}
	\item $P(h \mid d)$ Η πιθανότητα μιας υπόθεσης δεδομένων των παραδειγμάτων. Την αποκαλούμε εκ των υστέρων πιθανότητα, καθώς την υπολογίζουμε αφού έχουμε δει τα δεδομένα.
	\item $P(d \mid h)$ Η πιθανότητα να έχω τα παραδείγματα d, δεδομένου του ότι η υπόθεση h είναι σωστή.
	\item $P( h)$ Η πιθανότητα η υπόθεση h να είναι σωστή. Ονομάζεται εκ των προτέρων πιθανότητα, αφού την υπολογίζουμε βασιζόμενοι σε κάποια πεποίθηση και χωρίς κάποια γνώση για τα δεδομένα.
	\item $P(d)$ Η πιθανότητα των δεδομένων. Θα δούμε στη συνέχεια πως δεν χρειάζεται να ασχοληθούμε μαζί της.
\end{itemize}
\paragraph{Υπολογισμός Μοντέλου} Η διαδικασία εφαρμογής του αλγορίθμου αυτού είναι η εξής: αρχικά, έχοντας τα χαρακτηριστικά και την κλάση κάθε παραδείγματος στο σετ εκπαίδευσης, υπολογίζουμε την πιθανότητα κάθε κλάσης, ως τη συχνότητα εμφάνισής της. Στη συνέχεια, υπολογίζουμε τις πιθανότητες κάθε τιμής ενός χαρακτηριστικού. Αν για παράδειγμα προβλέπουμε την πιθανότητα να βρέξει $(rain=’yes’)$ με βάση την ύπαρξη σύννεφων$(cloudy=’yes’)$, τότε υπολογίζουμε:
$$P(cloudy= 'yes' \mid rain='yes')= \frac{count(cloudy='yes' , rain ='yes')}{count(rain='yes')}$$
\paragraph{Πρόβλεψη} Όταν φτάσει κάποιο στοιχείο για το οποίο θέλουμε να προβλέψουμε την κλάση του, τότε χρησιμοποιούμε το Θεώρημα Bayes για να υπολογίσουμε την πιθανότητα κάθε κλάσης και να διαλέξουμε την μεγαλύτερη. Σε αυτό το σημείο παρατηρούμε πως η ποσότητα $P(d)$ στον παρονομαστή είναι σταθερή
για κάθε κλάση και επομένως δεν συνεισφέρει στον υπολογισμό της πιθανότερης υπόθεσης. Άρα αρκεί να μεγιστοποιήσουμε την ποσότητα:
$$MAP(h)=P(d \mid h) P(h)$$
Μερικές παρατηρήσεις σχετικά με αυτόν τον αλγόριθμο:
\begin{itemize}
	\item ο υποτιμητικός χαρακτηρισμός του ως ”απλοϊκό”, οφείλεται στην υπόθεση του θεωρήματος Bayes για στατιστική ανεξαρτησία των γεγονότων. Αν και στα περισσότερα πραγματικά προβλήματα δεν ικανοποιείται μια τέτοια απαίτηση για τα χαρακτηριστικά των δεδομένων, ο αλγόριθμος αυτός συνεχίζει να δίνει καλά αποτελέσματα, διαψεύδοντας το όνομά του.
	\item καθώς στον υπολογισμό κάποιων πιθανοτήτων εμπλέκεται πολλαπλασιασμός πολλών και δυνητικά μικρών πιθανοτήτων, υπάρχει ο κίνδυνος μαθηματικής υποροής στο λογισμικό που τις εκτελεί. Για αυτό το λόγο συνηθίζουμε να δουλεύουμε με τους λογαρίθμους των πιθανοτήτων και όχι απευθείας με τις πιθανότητες.
	\item μόλις έχουμε ολοκληρώσει μια πρόβλεψη και είμαστε σίγουροι για αυτήν, μπορούμε να επαναυπολογίσουμε το μοντέλο για να το εμπλουτίσουμε με τη νέα γνώση.
\end{itemize}