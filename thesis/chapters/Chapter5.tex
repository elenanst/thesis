\chapter{Σχετική Δουλειά}
Η βιβλιογραφία αντικατοπτρίζει την προσπάθεια της κοινότητας να αυτοματοποιήσει τη διαδικασία της μηχανικής μάθησης προσφέροντας πακέτα λογισμικού που την υλοποιούν και έρευνες που επιχειρούν να επεκτείνουν το state-of-the-art. Οι εξελίξεις εντοπίζονται σε τομείς όπως η βελτιστοποίηση υπερ-παραμέτρων, η εισαγωγή μετα-μάθησης και η ανάπτυξη καλύτερων τεχνικών σχηματισμού πολύπλοκων μοντέλων.

Η ανατροπή του τοπίου για τη βελτιστοποίηση υπερ-παραμέτρων συνέβη όταν οι \citet{Bergstra:2012:RSH:2188385.2188395} απέδειξαν πως η τεχνική της πλεγματικής αναζήτησης επιφέρει αποτέλεσμα χειρότερο από την τυχαία αναζήτηση. Έκτοτε έχουν δοκιμαστεί γενετικοί αλγόριθμοι \citep{1554741}, αναζήτηση κλίσης \citep{wassenberg} και η bayesian βελτιστοποίηση, η οποία φαίνεται να έχει επικρατήσει με τη μορφή της τεχνικής SMBO \citep{DBLP:journals/corr/abs-1208-3719}. Οι \citet{HutHooLeyMur10} εισάγουν την έννοια των χρονικών ορίων στη διαδικασία της βελτιστοποίησης λαμβάνοντας υπόψιν ως κόστος τόσο την ποιότητα όσο και το χρόνο.

Προσπάθειες εισαγωγής μετα-μάθησης κατέβαλαν οι \citet{Feurer:2014:UMI:3015544.3015549}, οι οποίοι χρησιμοποίησαν μετα-χαρακτηριστικά των σετ δεδομένων, ώστε να προβλέψουν τιμές των υπερ-παραμέτρων που, με βάση παλαιότερα πειράματα, πιθανώς να οδηγούν σε καλύτερα μοντέλα. Ωστόσο η διαπίστωση αδυναμίας ικανοποιητικής πρόβλεψης τους οδήγησε σε χρήση των τιμών αυτών για αρχικοποίηση του SMBO αλγόριθμου αναζήτησης που χρησιμοποιούν.  

Οι \citet{kuba2002exploiting} χρησιμοποιούν δέντρα παλινδρόμησης για να προβλέψουν τις παράμετρους $\epsilon$ και $\sigma$ ενός \gls{SVM}. Έπειτα από μία διεξοδική ανάλυση των μετα-χαρακτηριστικών καταλήγουν σε μη ικανοποιητικά μοντέλα πρόβλεψης, πρόβλημα που προτείνουν να διορθώσουν εισάγοντας ένα τελικό στάδιο τοπικής αναζήτησης γύρω από τις προβλέψεις τους.


Οι \citet{Soares2004} ασχολούνται με τη πρόβλεψη της υπερ-παραμέτρου $\sigma$ ενός SVM, που καθορίζει το πλάτος του γκαουσιανού πυρήνα. Χρησιμοποιώντας μετα-χα\-ρακτηριστικά των σετ δεδομένων και ένα μοντέλο k-κοντινότερου γείτονα προβλέπουν τη διάταξη προκαθορισμένων τιμών της υπερ-παραμέτρου και με τη τεχνική της Top-Ν αξιολόγησης επιλέγουν τις βέλτιστες τιμές. Το σύστημά τους δε προβλέπει άμεσα τη βέλτιστη υπερ-παράμετρο, αλλά κατατάσσει ένα προκαθορισμένο σετ ως προς την απόδοσή του στο νέο σετ δεδομένων. Η μεθοδολογία τους απαιτεί τον προ-υπολογισμό της απόδοσης του SVM στα σετ δεδομένων εκπαίδευσης για τις διαθέσιμες τιμές, προσέγγιση απαγορευτική για πολυδιάστατους αλγορίθμους μάθησης. Επίσης, η ελευθερία επιλογής του Ν αυξάνει την απόδοση, αλλά καθιστά μια σχεδιαστική επιλογή, η οποία μειώνει τον αυτοματισμό της διαδικασίας. Τέλος, η μέθοδός τους εξασφαλίζει χειρότερο αποτέλεσμα από αυτό που επιτυγχάνεται με τη τεχνική Cross-validation, ωστόσο κρίνεται ικανοποιητική καθώς επιφέρει χρονική και υπολογιστική βελτίωση. 

Ενδιαφέρον παρουσιάζουν οι προσπάθειες των ερευνητών να αναλύσουν τη διαδικασία της μετα-μάθησης, ώστε να ανακαλύψουν τους μηχανισμούς που τη διέπουν με στόχο την αναγνώριση χρήσιμων χαρακτηριστικών, κατάλληλων αλγορίθμων μετα-μάθησης και γενικότερα την παραγωγή μετα-γνώσης. Οι \citet{Brazdil2009} τοποθετούν τη μετα-μάθηση μέσα στον τομέα της Εξόρυξης Δεδομένων, την προσδιορίζουν ως την ικανότητα προσαρμογής με βάση προϋπάρχουσα εμπειρία, παραθέτουν την ιστορική της εξέλιξη και παρουσιάζουν συστήματα που τη χρησιμοποιούν.

Εκτεταμένη έρευνα πάνω στη χρήση \gls{IBL} αλγορίθμων για πρόβλεψη υπερ-παραμέτρων πραγματοποιούν οι \citet{Abdulrahman:2014:MCA:3015544.3015557}. Αποδίδουν την καταλληλότητα των αλγορίθμων αυτών για μοντέλα μετα-μάθησης στην εκ φύσεως αδυναμία του προβλήματος για δημιουργία γενικών μοντέλων λόγω των περιορισμένων δεδομένων και της ιδιαιτερότητας κάποιων υπερ-παραμέτρων και τη δυνατότητά τους να ενημερώνονται χωρίς εκπαίδευση. Επίσης, παροτρύνουν προς την επιλογή μετα-χαρακτηριστικών με βάση τη σημασιολογική τους συνεισφορά στο πρόβλημα και την αξιολόγησή τους ως προς τη συσχέτισή τους με την υπερ-παράμετρο υπό πρόβλεψη. 

Οι \citet{Reif_meta2-features:} αναγνωρίζουν το πρόβλημα που προκύπτει στη προσπάθεια εφαρμογής μετα-μάθησης όταν τα μετα-χαρακτηριστικά έχουν διαφορετικό πλήθος για τα σετ δεδομένων και εισάγουν την έννοια των μετα-μετα-χαρακτηριστικών. Πρόκειται για μια προσπάθεια στατιστικής περιγραφής των μετα-χαρακτηριστικών μέσω στατιστικών δεικτών όπως η μέση τιμή, η διακύμανση και η κυρτότητα. Η τεχνική τους, πέρα από τη λύση του προηγούμενου προβλήματος, εμπλουτίζει τη πληροφορία που διατίθεται για μετα-μάθηση. Οι \citet{Bensusan:2001:EPA:645328.650030} εισάγουν τη χρήση πληροφορίας προερχόμενης από τα ιστογράμματα των μετα-χαρακτηριστικών προκειμένου να αποτυπώσουν πληρέστερα την κατανομή τους.

Η συνειδητοποίηση ότι η επίλυση προβλημάτων με χρήση αυτοματοποιημένων συστημάτων απαιτεί τη χρήση πληθώρας τεχνικών και αλγορίθμων οδήγησε στην αναζήτηση βελτιστοποιημένων μεθόδων συνδυασμού μοντέλων μάθησης. Οι \citet{Caruana:2004:ESL:1015330.1015432} εισάγουν τη μέθοδο σχηματισμού ensembles μοντέλων με την τεχνική της προς-τα-εμπρός επιλογής αναπτύσσοντας τεχνικές για την εξασφάλιση της καλής ποιότητας και την αποφυγή υπερ-προσαρμογής του τελικού ensemble. Διαφορετική προσέγγιση ακολοθούν οι \citet{ensemble}, οι οποίοι απορρίπτουν το σχηματισμό του ensemble ως τελικό στάδιο. Η προσέγγισή τους ενσωματώνει το σχηματισμό στη διαδικασία της βελτιστοποίησης των υπερ-παραμέτρων καθώς η συνάρτηση κόστους υπό βελτιστοποίηση αφορά τη συνεισφορά ενός μοντέλου στον ensemble και η εισαγωγή ενός νέου μοντέλου στη συλλογή γίνεται με μία round robin στρατηγική. 