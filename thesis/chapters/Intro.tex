\begin{titlepage}
	
\begin{tabular}{ll}
	\includegraphics[scale=0.13]{university} &\makecell[l]{ \small Αριστοτέλειο Πανεπιστήμιο Θεσσαλονίκης\\Πολυτεχνική Σχολή\\Τμήμα Ηλεκτρολόγων Μηχανικών και Μηχανικών Υπολογιστών\\Εργαστήριο Επεξεργασίας Πληροφορίας και Υπολογισμών\\
		\\
		\\ \\}
\end{tabular}
\begin{center}	

			
		\vspace{1cm}
		
		\LARGE
		Διπλωματική\\
		Εργασία		
		\vspace{0.3cm}
		
		\Huge
		\textbf{Αυτοματοποιημένος Αναλυτής Δεδομένων}
		
		\vspace{0.3cm}
		\LARGE
		Σχεδίαση και Υλοποίηση ενός Συστήματος Αυτόματης Εκπαίδευσης Μοντέλων Μηχανικής Μάθησης
		
		\vspace{1.5cm}
		
		Εκπόνηση \\
		\textbf{Νησιώτη Ελένη 7737}
		
		\vspace{1.5cm}
		
		Επίβλεψη \\
		\textbf{Επίκ. Καθ. Ανδρέας Συμεωνίδης}
		
		Συνεπίβλεψη \\
		\textbf{Δρ. Κυριάκος Χατζηδημητρίου}
		
		\vspace{1cm}
					
		\today

		
	\end{center}
\end{titlepage}

{
	\titleformat{\paragraph}[display]{\normalfont\Large\bfseries\centering}{\theparagraph}{1em}{}
	\paragraph{Περίληψη}
	Η επιστήμη της μηχανικής μάθησης έχει κατορθώσει, βασιζόμενη σε αυστηρά μαθηματικά εργαλεία, να μετατρέψει τη σύγχρονη αφθονία πληροφορίας σε κατανόηση κοινωνικών, οικονομικών και φυσικών μηχανισμών, γεγονός που οδήγησε στην εκτεταμένη δημιουργία προβλεπτικών μοντέλων. Η εξέλιξή της, ωστόσο, περιορίζεται σήμερα από την ύπαρξη απαιτητικών προβλημάτων και την εγγενή αδυναμία εφαρμογής μια δοκιμασμένης μεθοδολογίας των πειραμάτων και μοντέλων σε νέα προβλήματα. Ανέκυψε λοιπόν η ανάγκη ανακάλυψης μονοπατιών που θα οδηγήσουν σε βαθύτερη κατανόηση των μηχανισμών της μηχανικής μάθησης, ώστε πλέον να εκπαιδεύουμε μοντέλα που βελτιστοποιούν την ίδια τη διαδικασία της μάθησης, και όχι μεμονωμένα προβλήματα. Το πεδίο του ΑutoML αναδύθηκε πρόσφατα από αυτήν την προσπάθεια και σύγχρονοι ορισμοί του αποδίδουν την αυτοματοποίηση της εφαρμογής μηχανικής μάθησης. Συναντάται κυρίως με τη μορφή εργαλείων λογισμικού, τα οποία υποβοηθούν τον αναλυτή δεδομένων, αναλαμβάνοντας την αναζήτηση αυτόματων λύσεων που καθιστούν την εφαρμογή μηχανικής μάθησης πιο αποδοτική και αποτελεσματική. Χαρακτηριστικό αυτών των συστημάτων είναι η ύπαρξη μετα-γνώσης, δηλαδή γνώσης προερχόμενης από την εφαρμογή μηχανικής μάθησης σε παρελθοντικά προβλήματα, η οποία προσδίδει εμπειρία και προσαρμοστικότητα στο σύστημα. Την υλοποίηση ενός τέτοιου εργαλείου έχει ως στόχο η παρούσα διπλωματική εργασία, καθώς αναγνωρίζει την ανάγκη έρευνας και επέκτασης των εφαρμογών του \gls{AutoML}. Εκμεταλλευόμενοι σύγχρονες τεχνολογίες, όπως την πολυπληθή αποθήκη πακέτων της γλώσσας R, εξερευνήσαμε τις δυνατότητες τεχνικών μηχανικής μάθησης και επιχειρήσαμε να επεκτείνουμε την τρέχουσα κατάσταση ενσωματώνοντας στο σύστημά μας μετα-μάθηση για τη βελτιστοποίηση υπερ-παραμέτρων και ensembles με προς τα εμπρός επιλογή μοντέλων. Βασικός στόχος της εργασίας μας είναι η σχεδίαση και υλοποίηση ενός έμπειρου, κατανοητού και επεκτάσιμου αυτόματου αναλυτή δεδομένων. Θεωρούμε πως η βιβλιογραφική έρευνα, τα πειράματα και το εργαλείο λογισμικού που υλοποιήθηκαν κατά τη διάρκεια της δουλειάς μας μπορούν να αποτελέσουν σημαντική συνεισφορά στο πεδίο του \gls{AutoML}. 
	\newpage
	\begin{center}
		\textbf{\huge{Diploma Thesis\\}}
		\vspace{0.5cm}
		\textbf{\Large{Automated Data Scientist}}
	\end{center}
	
	\paragraph{Abstract}
	The science of machine learning has achieved, based on solid mathematical tools, to transform the current data deluge into the under\-standing of underlying social, economical and nature mechanisms and the generation of related predictive models. However, the presence of computa\-tionally demanding problems and the current inability to automatically transfer the knowledge on how to apply machine learning on new applications and new problems, delays the evolution of knowledge itself. The necessity of discovering paths that lead to a deeper understanding of the machine learning mecha\-nisms is evident, bearing the ambition of training models that optimize the very process of learning, instead of individual applications. \gls{AutoML}, that has recently emerged, attempts to automate the application of machine learning. Its most apparent manifestations  include software systems that serve as productivity tools, instruments to make experts more efficient and effective, but not eliminate them. A common feature of these systems is the embedding of meta-knowledge, namely knowledge produced by the application of machine learning in past experi\-ments, a trait that adds experience and adaptability to the system. This diploma thesis aims at implementing a software tool to facilitate the AutoML process. Exploiting current technologies, such as the rich CRAN repository, we explored opportunities offered by machine learning techni\-ques and have attempted to push forward the state of the art by embedding meta-learning for optimal hyperparameter selection and forward model selection ensembles to our system. Main aspiration of our work consisted in designing and implementing an experien\-ced, intuitive and expandable automated data analyst. The experiments  seem promising, and we argue that the implemented tool could constitute an informa\-tive contribu\-tion to the area of \gls{AutoML}.       
	
	\vspace{2cm}
	
	\begin{flushleft}
		\textit{Eleni Nisioti \\
		Intelligent Systems \& Software Engineerin Labgroup\\
		Electrical \& Computer Engineering Department\\
		Aristotle University of Thessaloniki\\
		June 2017}
	\end{flushleft}
	
	\newpage
	\paragraph{Ευχαριστίες}
	Η εκπόνηση της διπλωματικής εργασίας ήταν μια γεμάτη και διδακτική εμπειρία που συμπεριέλαβε άτομα, τα οποία θα ήθελα να ευχαριστήσω σε αυτό το σημείο.
	
	Τον επίκουρο καθηγητή κ. Ανδρέα Συμεωνίδη για την εμπιστοσύνη  που μου έδειξε με την ανάθεση της διπλωματικής εργασίας, την κατανόηση που επέδειξε μέχρι το πέρας της και την ακαδημαϊκή στήριξή του.
	
	Τον μεταδιδακτορικό ερευνητή κ. Κυριάκο Χατζηδημητρίου για την καθοδήγηση, την εμπιστοσύνη του στην κρίση μου και την εισαγωγή μου σε ένα ανεξερεύνητο επιστημονικό αντικείμενο.
	
	Την οικογένειά μου για τη στήριξη και την αλόγιστη εμπιστοσύνη της στις αποφάσεις μου.
	
	Τους φίλους μου για τις εμπειρίες των τελευταίων 6 χρόνων.
	\newpage
}


\tableofcontents
\listoffigures
\listoftables

\pagebreak
\thispagestyle{empty}
\hspace{0pt}
\vfill
\begin{flushright}
"\textit{When you have eliminated the impossible, whatever remains, no matter how improbable, must be the truth.}"
\\[8pt]
\rightline{{\rm --- The Sign of Four}}
\end{flushright}
\vfill
\hspace{0pt}
\pagebreak




