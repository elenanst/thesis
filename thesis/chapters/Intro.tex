\begin{titlepage}
			\begin{minipage}{0.3\textwidth}
				\begin{flushright}
				\includegraphics[scale=0.15]{university}
			\end{flushright}
			\end{minipage} \quad
		\begin{minipage}{0.7\textwidth}
				\small
				Αριστοτέλειο Πανεπιστήμιο Θεσσαλονίκης\\
				Πολυτεχνική Σχολή\\
				Τμήμα Ηλεκτρολόγων Μηχανικών και Μηχανικών Υπολογιστών\\
				Εργαστήριο Επεξεργασίας Πληροφορίας και Υπολογισμών\\
		\end{minipage}
	\begin{center}	

			
		\vspace{1.5cm}
		
		\LARGE
		Διπλωματική\\
		Εργασία		
		\vspace{0.3cm}
		
		\Huge
		\textbf{Αυτοματοποιημένος Αναλυτής Δεδομένων}
		
		\vspace{0.3cm}
		\LARGE
		Σχεδίαση και Υλοποίηση ενός Συστήματος Αυτόματης Εκπαίδευσης Μοντέλων Μηχανικής Μάθησης
		
		\vspace{1.5cm}
		
		Εκπόνηση \\
		\textbf{Νησιώτη Ελένη}
		
		\vspace{1.5cm}
		
		Επίβλεψη \\
		\textbf{Επίκ. Καθ. Ανδρέας Συμεωνίδης}
		
		Συνεπίβλεψη \\
		\textbf{Δρ. Χατζηδημητρίου Κυριάκος}
		
		\vspace{1cm}
					
		\today

		
	\end{center}
\end{titlepage}

{
	\titleformat{\paragraph}[display]{\normalfont\Large\bfseries\centering}{\theparagraph}{1em}{}
	\paragraph{Περίληψη}
	Η επιστήμη της μηχανικής μάθησης έχει κατορθώσει, βασιζόμενη σε αυστηρά μαθηματικά εργαλεία, να μετατρέψει τη σύγχρονη πληροφοριακή αφθονία σε κατανόηση κοινωνικών, οικονομικών και φυσικών μηχανισμών, γεγονός που οδήγησε στην εκτεταμένη δημιουργία προβλεπτικών μοντέλων. Η εξέλιξή της, ωστόσο, περιορίζεται σήμερα από την ύπαρξη απαιτητικών προβλημάτων και την εγγενή αδυναμία μεταφερσιμότητας των πειραμάτων και μοντέλων σε νέα προβλήματα. Ανέκυψε λοιπόν η ανάγκη ανακάλυψης μονοπατιών που θα οδηγήσουν σε βαθύτερη κατανόηση των μηχανισμών της μηχανικής μάθησης, ώστε πλέον να εκπαιδεύουμε μοντέλα που βελτιστοποιούν την ίδια τη διαδικασία της μάθησης, και όχι μεμονωμένα προβλήματα. Το πεδίο του ΑutoML αναδύθηκε πρόσφατα από αυτήν την προσπάθεια και σύγχρονοι ορισμοί του αποδίδουν την αυτοματοποίηση της εφαρμογής μηχανικής μάθησης. Συναντάται κυρίως με τη μορφή εργαλείων λογισμικού, τα οποία υποβοηθούν τον αναλυτή δεδομένων, αναλαμβάνοντας την αναζήτηση αυτόματων λύσεων που καθιστούν την εφαρμογή μηχανικής μάθησης πιο αποδοτική και αποτελεσματική. Χαρακτηριστικό αυτών των συστημάτων είναι η ύπαρξη μετα-γνώσης, δηλαδή γνώσης προερχόμενης από την εφαρμογή μηχανικής μάθησης σε παρελθοντικά προβλήματα, η οποία προσδίδει εμπειρία και προσαρμοστικότητα στο σύστημα. Την υλοποίηση ενός τέτοιου εργαλείου έχει ως στόχο η παρούσα διπλωματική εργασία, καθώς αναγνωρίζει την ανάγκη έρευνας και επέκτασης των εφαρμογών του \gls{AutoML}. Εκμεταλλευόμενοι σύγχρονες τεχνολογίες, όπως την πολυπληθή αποθήκη πακέτων της γλώσσας R, εξερευνήσαμε τις δυνατότητες τεχνικών μηχανικής μάθησης και επιχειρήσαμε να επεκτείνουμε την τρέχουσα κατάσταση ενσωματώνοντας στο σύστημά μας μετα-μάθηση για τη βελτιστοποίηση υπερ-παραμέτρων και ensembles με προς-τα-εμπρός επιλογή μοντέλων. Κυρίαρχο στόχο της εργασίας μας αποτέλεσε η σχεδίαση και υλοποίηση ενός έμπειρου, κατανοητού και επεκτάσιμου αυτόματου αναλυτή δεδομένων. Θεωρούμε πως η βιβλιογραφική έρευνα, τα πειράματα και το εργαλείο λογισμικού που υλοποιήθηκαν κατά τη διάρκεια της δουλειάς μας αποτελούν σημαντική συνεισφορά στο πεδίο του \gls{AutoML}. 
	\newpage
	\begin{center}
		\textbf{\huge{Diploma Thesis\\}}
		\vspace{0.5cm}
		\textbf{\Large{Automated Data Scientist}}
	\end{center}
	
	\paragraph{Abstract}
	The science of machine learning has achieved, based on solid mathematical tools, to convert the current informational abundance into the under\-standing of social, economical and nature mechanisms that lead to the general creation of predictive models. Its evolution, however, has stumbled upon  the presence of computationally demanding problems and an inherent lack of transferability of machine learning experiments to new applications. The necessity, therefore, came up of discovering paths that lead to a deeper understanding of the machine learning mecha\-nismus, with the ambition of training models that optimize the very process of learning, instead of individual applications. The field of \gls{AutoML} emerged recently through this attempt and contempo\-rary definitions acknowledge to it the automation of applying machine learning. Its most apparent manifestations  include software systems that serve as productivity tools, instruments to make experts more efficient and effective, but not eliminate them. A common feature of these systems is the embedding of meta-knowledge, namely knowledge produced by the application of machine learning in past experi\-ments, a trait that adds experience and adaptability to the system. This Diploma Thesis aims at the implementation of a software tool belonging to the above described family, an idea sparked from the realization of \gls{AutoML}'s need for research and enrichment of its applications. Exploiting current technologies, susch as the rich CRAN repository, we explored opportunities offered by machine learning techniques and attempted to push forward the state of the art by embedding meta-learning for optimal hyperparameter selection and forward-model-selection ensembles to our system. Main aspiration of our work consisted in designing and implementing an experien\-ced, intuitive and expandable automated data analyst. We deem that the acade\-mic research, experiments and software produced during our work constitute an informa\-tive contribu\-tion to the area of \gls{AutoML}.       
	
	\vspace{2cm}
	
	\begin{flushleft}
		\textit{Eleni Nisioti \\
		Intelligent Systems \& Software Engineerin Labgroup\\
		Electrical \& Computer Engineering Department\\
		Aristotle University of Thessaloniki\\
		June 2017}
	\end{flushleft}
	
	\newpage
	\paragraph{Ευχαριστίες}
	\newpage
}


\tableofcontents
\listoffigures
\listoftables

\pagebreak
\thispagestyle{empty}
\hspace{0pt}
\vfill
\begin{flushright}
"\textit{When you have eliminated the impossible, whatever remains, no matter how improbable, must be the truth.}"
\\[8pt]
\rightline{{\rm --- The Sign of Four}}
\end{flushright}
\vfill
\hspace{0pt}
\pagebreak




