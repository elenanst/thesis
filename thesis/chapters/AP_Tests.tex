\chapter{Στατιστικά τεστ Υπόθεσης}
\label{appendix:Tests}
Γενικά χαρακτηριστικά των στατιστικών τεστ υπόθεσης έχουν περιγραφεί στην Ενότητα \ref{section:tests}. Σε αυτό το σημείο θα αναφέρουμε περιληπτικά μερικά είδη τέτοιων τεστ, τα οποία διαφοροποιούνται κυρίως ως προς:
\begin{itemize}
	\item τις υποθέσεις που κάνουν για τους πληθυσμούς.
	\item τη στατιστική που χρησιμοποιούν για να περιγράψουν το δείγμα.
\end{itemize}
\paragraph{Pearson's Chi-squared τεστ}
Πρόκειται για ένα στατιστικό τεστ μεταξύ δύο συνόλων κατηγορικών δεδομένων που εξετάζει αν οι διαφορές τους προκλήθηκαν τυχαία. Είναι κατάλληλο για μη-ζευγαρωμένα (unpaired) δεδομένα από μεγάλα δείγματα. Προέρχεται από την ευρύτερη οικογένεια των τεστ που αξιολογούνται με αναφορά στην κατανομή chi-squared, για την οποία όταν η μηδενική υπόθεση είναι αληθής η κατανομή του test statistic είναι chi-squared~\footnote{https://en.wikipedia.org/wiki/Chi-squared\_test}. 
\paragraph{Yate's correction for continuity}
Η τεχνική αυτή χρησιμοποιείται για διόρθωση του εξής προβλήματος: κατά την εφαρμογή του Pearson's chi-squared τεστ γίνεται η υπόθεση πως η διακριτή πιθανότητα των παρατηρούμενων συχνοτήτων στον πίνακα ενδεχομένων μπορεί να προσεγγιστεί από μία συνεχή chi-squared κατανομή.
\paragraph{ANOVA τεστ} Εισήχθη στο \citet{QJ:QJ49708235130} ως μία τεχνική ανάλυσης των διαφορών που παρουσιάζονται στις μέσες τιμές διαφορετικών ομάδων. Στην περίπτωση που οι ομάδες είναι ανεξάρτητες χρησιμοποιείται η one-way εκδοχή, ενώ όταν υπάρχει κάποια συσχέτιση μεταξύ τους η repeated-measures. Το τεστ αυτό χρησιμοποιείται για την περίπτωση σύγκρισης περισσότερων των τριών πληθυσμών, καθώς πολλαπλά t-tests θα οδηγούσαν σε μη αποδεκτό σφάλμα τύπου Ι. 

Προκειμένου να ορίσει το F-statistic η τεχνική αυτή αναλύει τη διακύμανση που εμφανίζεται στο πληθυσμό σε αυτή που οφείλεται σε διαφορές μεταξύ των διαφορετικών ομάδων και διαφορές εντός των ομάδων, δηλαδή διαχωρίζει τις πηγές διακύμανσης. Οι υποθέσεις που κάνει αυτό το τεστ είναι:
\begin{itemize}
	\item Κανονική κατανομή της εξαρτημένης μεταβλητής για κάθε ομάδα.
	\item Υπάρχει ομοιογένεια στις διακυμάνσεις, δηλαδή είναι ίσες για κάθε ομάδα.
	\item Οι παρατηρήσεις είναι ανεξάρτητες, γεγονός που καθορίζεται κατά τη συλλογή των δεδομένων.
\end{itemize}

\paragraph{Friedman τεστ}
Πρόκειται για ένα μη-παραμετρικό τεστ για την ανίχνευση διαφορών μεταξύ πολλών αλγορίθμων σε πολλά σετ δεδομένων. Θεωρείται μια μη-παραμετρική εκδοχή της ANOVA, με απόρροια την απεμπλοκή από τις υποθέσεις της κανονικής κατανομής, των ίσων διακυμάνσεων των residuals και την απώλεια ισχύος.  

Σημαντική προσθήκη αποτελεί η εναλλακτική test statistic που εισήγαγαν οι \citet{doi:10.1080/03610928008827904}, καθώς διαπίστωσαν ότι η βασική ήταν ανεπιθύμητα συντηρητική. 

Σε περίπτωση διαπίστωσης σημαντικής στατιστικής διαφοράς στην απόδοση πολλών αλγορίθμων προκύπτει η ανάγκη εξακρίβωσης των ζευγαριών που οδήγησαν σε αυτό το αποτέλεσμα. Προς αυτό το σκοπό μπορούν να χρησιμοποιηθούν τα εξής post-hoc τεστ: η διαδικασία Tukey, το Dunnett τεστ, η διόρθωση Bonferroni, το τεστ Nemenyi, η προς-τα-κάτω διαδικασία του Holm, η διαδικασία του Hommel \citep{CIS-57999, citeulike:4294367, doi:10.1093/biomet/75.2.383}.
\paragraph{Fisher's exact τεστ}
 Εισήχθη από τον Fisher μέσω του παραδείγματος της Κυρίας που δοκιμάζει Τσάι~\footnote{https://en.wikipedia.org/wiki/Lady\_tasting\_tea} ως ένα τεστ για κατηγορικά δεδομένα, τα οποία κατατάσσονται μεταξύ δύο κατηγοριών. Σύμφωνα με την ανάλυση η μηδενική υπόθεση αντιστοιχεί σε υπερ-γεωμετρική κατανομή των δεδομένων.
  
 Χρησιμοποιείται κυρίως στην περίπτωση μικρών δειγμάτων. Λέγεται ακριβές επειδή για μικρά δείγματα η σημασία της διακύμανσης από τη μηδενική υπόθεση (p-value) μπορεί να υπολογιστεί ακριβώς αντί να βασίζεται σε μια προσέγγιση που γίνεται ακριβής καθώς το μέγεθος του δείγματος πλησιάζει το άπειρο.
\paragraph{Mann–Whitney U τεστ (Wilcoxon rank-sum)} Εισήχθη από τον \citet{Wilcoxon45} και αναλύθηκε διεξοδικά από τους \citet{mann1947}. Πρόκειται για ένα μη-παραμετρικό τεστ της μηδενικής υπόθεσης ότι είναι εξίσου πιθανό μία τυχαία επιλεγμένη τιμή από ένα δείγμα να είναι μικρότερη ή μεγαλύτερη από μία επιλεγμένη τιμή από ένα άλλο δείγμα. Σε αντίθεση με το t-test δεν απαιτεί κανονικότητα των πληθυσμών.
 
\paragraph{McNemar}
Εισήχθη από τον \citet{McNemar1947} ως ένα τεστ για ζευγαρωμένα ονομαστικά δεδομένα, δηλαδή δεδομένα που διαφοροποιούνται μόνο από το όνομά τους και υπάρχει ένα-προς-ένα συσχέτιση μεταξύ τους.

\paragraph{Cochran-Mantel-Haenszel} Συνδιαμορφώθηκε από τους \citet{10.2307/3001616, doi:10.1093/jnci/22.4.719} και αποτελεί γενίκευση του McNemar, καθώς υποστηρίζει διαστρωμάτωση των δεδομένων σε αυθαίρετο πλήθος ομάδων.