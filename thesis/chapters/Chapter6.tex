\chapter{Σύνοψη}
Στη διάρκεια της εκπόνησης της διπλωματικής εργασίας φτάσαμε σε συμπεράσματα, τα οποία εκφράζονται με ουσιώδεις διαπιστώσεις για το αντικείμενο, εγγενείς δυσκολίες του προς επίλυση προβλήματος και μία κριτική ενατένιση της σύγχρονης βιβλιογραφίας.

Διαπιστώσαμε πως η σχεδίαση ενός ολοκληρωμένου εργαλείου \gls{AutoML} απαιτεί πολύπλευρη προσέγγιση, καθώς οφείλει να επιτελεί λειτουργίες για:
\begin{itemize}
	\item \textbf{Την αυτοματοποίηση τετριμμένων καθηκόντων}
	\item \textbf{Την προσαρμοζόμενη αντιμετώπιση κάθε προβλήματος}
	\item \textbf{Την ενσωμάτωση ποικίλλων τεχνικών μηχανικής μάθησης, γεγονός που απαιτεί}
	\item \textbf{Τη χρήση ensembles με ισχυρούς μηχανισμούς επιλογής και αξιολόγησης μοντέλων}
\end{itemize}

Εντοπίσαμε δυσκολίες κατά την εφαρμογή μετα-μάθησης, καθώς τα μετα-μοντέλα αποδείχτηκαν ανίσχυρα και η ενίσχυσή τους με διαστήματα πρόβλεψης χρονικά και υπολογιστικά απαιτητική. Το γεγονός αυτό ήταν βέβαια αναμενόμενο δεδομένης της νεότητας αυτού του αντικειμένου και των διαπιστωμένων, εγγενών δυσκολιών στη σχετική βιβλιογραφία \citep{Feurer:2014:UMI:3015544.3015549, kuba2002exploiting, Soares2004}.

Η επαφή με το αντικείμενο του \gls{AutoML} μας αποκάλυψε τις αδυναμίες της σύγχρονης εφαρμογής μηχανικής μάθησης, που αιτιολογούν, αλλά δε δικαιολογούν, τη στασιμότητα του αντικειμένου: αφέλεια, εμμονή σε παρελθοντικές πανάκειες, αδυναμία μεταφερσιμότητας εφαρμοσμένης γνώσης, έλλειψη σχεδιασμού επαναπαράξιμων πειραμάτων και η αναμενόμενη απαξίωσή τους από την επιστημονική κοινότητα.

Θεωρούμε πως το \gls{AutoML} αποτελεί προϊόν της συνειδητοποίησης της κοινότητας και το πέρασμα σε ένα στάδιο μάθησης, όπου με ωριμότητα, επαγωγική σκέψη και διερευνητική διάθεση θα προσεγγίσουμε την εκμάθηση, και όχι εφαρμογή, της (μηχανικής) μάθησης.