\chapter{Σχετική Δουλειά}
Η βιβλιογραφία αντικατοπτρίζει την προσπάθεια της κοινότητας να αυτοματοποιήσει τη διαδικασία της μηχανικής μάθησης προσφέροντας πακέτα λογισμικού που την υλοποιούν και έρευνες που επιχειρούν να επεκτείνουν το state-of-the-art. Οι εξελίξεις εντοπίζονται σε τομείς όπως η βελτιστοποίηση υπερ-παραμέτρων, η εισαγωγή μετα-μάθησης και η ανάπτυξη καλύτερων τεχνικών σχηματισμού πολύπλοκων μοντέλων.

Η ανατροπή του τοπίου για τη βελτιστοποίηση υπερ-παραμέτρων συνέβη όταν οι \citet{} απέδειξαν πως η τεχνική της πλεγματικής αναζήτησης επιφέρει αποτέλεσμα χειρότερο από την τυχαία αναζήτηση. Έκτοτε έχουν δοκιμαστεί γενετικοί αλγόριθμοι \citep{1554741}, αναζήτηση κλίσης \citep{wassenberg} και η bayesian βελτιστοποίηση, η οποία φαίνεται να έχει επικρατήσει με τη μορφή της τεχνικής SMBO \citep{DBLP:journals/corr/abs-1208-3719}. Οι \citet{HutHooLeyMur10} εισάγουν την έννοια των χρονικών ορίων στη διαδικασία της βελτιστοποίησης λαμβάνοντας υπόψιν ως κόστος τόσο την ποιότητα όσο και το χρόνο. Προσπάθειες εισαγωγής μετα-μάθησης κατέβαλαν οι \citet{Feurer:2014:UMI:3015544.3015549}, οι οποίοι χρησιμοποίησαν μετα-χαρακτηριστικά των σετ δεδομένων, ώστε να αρχικοποιήσουν τον αλγόριθμο αναζήτησης που χρησιμοποιούν με πιθανά καταλληλότερες τιμές, ενώ οι \citet{Soares2004} εκπαιδεύουν έναν κ-κοντινότερο γείτονα με μετα-χαρακτηριστικά για να προβλέψουν το πλάτος του πυρήνα ενός SVM μοντέλου παλινδρόμησης.