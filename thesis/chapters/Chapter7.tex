\chapter{Συμπεράσματα}
Στη διάρκεια της εκπόνησης της διπλωματικής εργασίας φτάσαμε σε συμπεράσματα, τα οποία εκφράζονται με ουσιώδεις διαπιστώσεις για το αντικείμενο, εγγενείς δυσκολίες του προς επίλυση προβλήματος και μία κριτική ενατένιση της σύγχρονης βιβλιογραφίας.

Διαπιστώσαμε πως η σχεδίαση ενός ολοκληρωμένου εργαλείου \gls{AutoML} απαιτεί πολύπλευρη προσέγγιση, καθώς οφείλει να επιτελεί λειτουργίες για:
\begin{itemize}
	\item την αυτοματοποίηση τετριμμένων καθηκόντων
	\item την προσαρμοζόμενη αντιμετώπιση καε προβλήματος
	\item την ενσωμάτωση ποικίλλων τεχνικών μηχανικής μάθησης, γεγονός που απαιτεί 
	\item τη χρήση ensembles με ισχυρούς μηχανισμούς επιλογής και αξιολόγησης μοντέλων
\end{itemize}

Εντοπίσαμε δυσκολίες κατά την εφαρμογή μετα-μάθησης, καθώς τα μετα-μοντέλα αποδείχτηκαν ανίσχυρα και η ενίσχησή τους με διαστήματα πρόβλεψης χρονικά και υπολογιστικά απαιτητική. Το γεγονός αυτό ήταν βέβαια αναμενόμενο δεδομένης της νεότητας αυτού του αντικειμένου και των διαπιστωμένων, εγγενών δυσκολιών στη σχετική βιβλιογραφία \citep{Feurer:2014:UMI:3015544.3015549, kuba2002exploiting, Soares2004}.

Η επαφή με το αντικείμενο του \gls{AutoML} μας αποκάλυψε τις αδυναμίες της σύγχρονης εφαρμογής μηχανικής μάθησης, μας εισήγαγε στον τομέα της μετα-μάθησης και μας προσέφερε πρωτοποριακά εργαλεία λογισμικού. Αν και η αφέλεια, η εμμονή σε παρελθοντικές τεχνικές και η αδυναμία επαναχρησιμοποίησης, συνεπώς απουσία επουσιώδους χρησιμότητας, των μέχρι τώρα πειραμάτων φαινόταν να έχουν φέρει το αντικείμενο της μηχανικής μάθησης σε ένα τέλμα, κρίνουμε το μέλλον αισιόδοξα. Βάση αυτής της αισιοδοξίας αποτελεί το πέρασμα από την εφαρμογή της μηχανικής μάθησης στην εκμάθησή της, το \gls{AutoML}.