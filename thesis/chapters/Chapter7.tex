\chapter{Μελλοντικές Επεκτάσεις}
Στο σημείο αυτό θα παρουσιάσουμε ιδέες που προέκυψαν κατά την εκπόνηση της διπλωματικής, οι οποίες βελτιώνουν και επεκτείνουν το σύστημα που υλοποιήσαμε. Οι ιδέες αυτές ποικίλλουν ως προς την πολυπλοκότητα, την απαιτητικότητα και τη χρησιμότητά τους, αποτελούν ωστόσο προσθήκες χρήσιμες στο σύνολό τους, καθιστώντας το σύστημά μας πιο αποδοτικό, αυτόνομο και εύχρηστο.

Η θεωρία της μετα-μάθησης έχει δυνατότητα εφαρμογής σε οποιαδήποτε απόφαση επωφελείται από παρελθοντική γνώση. Το σύστημά μας ενσωματώνει τεχνικές μετα-μάθησης για την επιλογή βέλτιστων υπερ-παραμέτρων, ένα απαιτητικό, πολυδιάστατο πρόβλημα βελτιστοποίησης. Αναγνωρίζουμε τη δυνατότητα εφαρμογής μετα-μάθησης σε στάδια όπως η επιλογή αλγορίθμου μάθησης και τεχνικών προ-επεξεργασίας, λειτουργικότητες που θα προσέθεταν επιπλέον εμπειρία στο σύστημά μας. 

Επιπλέον επέκταση στο κομμάτι της μετα-μάθησης θα επιτευχθεί με την ενδεχόμενη βελτίωση των υπαρχόντων μετα-μοντέλων μέσω επιπλέον πειραμάτων. Τα νέα πειράματα μπορούν να εξερευνήσουν τη συνεισφορά:
\begin{itemize}
	\item \textbf{Επιπλέον μετα-χαρακτηριστικών} Προϋπόθεση αποτελεί η προσφορά νέων μετα-χαρα\-κτηριστικών από τη σύγχρονη βιβλιογραφία και η ανακάλυψη μετα-χαρακτηριστικών που δεν ανέδειξε η εκτεταμένη έρευνά μας.
	\item \textbf{Βελτίωση των διαστημάτων πρόβλεψης μέσω:}
	\begin{itemize}
		\item \textit{Πειραματισμού με διάφορα επίπεδα εμπιστοσύνης.} Το τρέχον σύστημα χρησιμοποιεί επίπεδο εμπιστοσύνης $95\%$, αύξηση του οποίου θα οδηγήσει πιθανότερα σε αποδοτικότερο μετα-μοντέλο, αλλά υψηλότερους χρόνους εκπαίδευσης.
		\item \textit{Πειραματισμού με διάφορα μεγέθη bootstrap δειγμάτων.} Το ανώτερο όριο, 50, έχει τεθεί λόγω χρονικής πολυπλοκότητας. Η διερεύνηση της βελτίωσης που θα προσέφερε η αύξηση των δειγμάτων για τον προσδιορισμό του ανώτατου ορίου, πέρα από το οποίο δεν προσφέρεται βελτίωση, θα είναι χρήσιμη στην περίπτωση που η χρονική πολυπλοκότητα δεν αποτελεί κριτήριο.
	\end{itemize} 
\end{itemize}

Ένα πείραμα μηχανικής μάθησης διαθέτει στάδια, τα οποία επιδέχονται παραλληλοποίηση, καθώς διασπώνται σε επιμέρους, ανεξάρτητα καθήκοντα. Παρόλο που ο χρονοβόρος σχηματισμός του τελικού ensemble παρουσιάζει συνέχεια που επιτάσσει την ολική αντιμετώπιση της αποθήκης μοντέλων σε κάθε επανάληψη, υπάρχουν στάδια ευκόλως παραλληλοποιήσιμα (embarassingly parallel), όπως η πρόβλεψη υπερ-παραμέτρων με χρήση μετα-μοντέλων, η αξιολόγηση με τις τεχνικές k-fold cross-validation και leave one out και η αποθήκευση των εκπαιδευμένων μοντέλων.

Ενδιαφέρουσα επέκταση του συστήματός μας αποτελεί η τροφοδότησή του με επιπλέον σετ δεδομένων. Προς αυτόν τον σκοπό δεν απαιτείται επέκταση της λειτουργικότητας του εργαλείου, αλλά συλλογή των σετ δεδομένων και επανεκπαίδευση των μετα-μοντέλων. Έτσι, το σύστημά μας θα αποκτήσει μεγαλύτερο πεδίο εφαρμογής.

Ευκολότερη επίτευξη του προ-αναφερόμενου στόχου και γενικότερη βελτίωση της ευχρηστίας του συστήματος θα επιφέρει η υλοποίηση και ενσωμάτωση λειτουργιών για την εκπαίδευση του συστήματος. Υποψήφιες λειτουργίες αποτελούν:
\begin{itemize}
	\item \textbf{Ένα εργαλείο αυτόματης συλλογής σετ δεδομένων} Κατά τη συλλογή των απαραίτητων σετ δεδομένων παρατηρήσαμε πως στη διαδικασία εμπλέκονται διαφορετικοί πάροχοι, οι οποίοι προσφέρουν σετ δεδομένων σε ποικίλες μορφές αρχείων και διαθέτουν περιγραφές τους που συχνά στερούνται χρήσιμης πληροφορίας. Ως αποτέλεσμα η συλλογή απαιτεί εκτεταμένη αναζήτηση, αξιολόγηση της πληροφορίας και καθαρισμό των αρχείων, στάδια που την καθιστούν χρονικά και νοητικά απαιτητική. Θεωρούμε πως ένα εργαλείο που θα λειτουργεί ως διεπαφή μεταξύ του συστήματός μας και των διαδικτυακών αποθηκών σετ δεδομένων θα προσφέρει ευχρηστία και θα βοηθήσει στην επέκταση του συστήματος.
	\item \textbf{Διεπαφή για εκπαίδευση μετα-μοντέλων} Η εκπαίδευση του συστήματος σε νέα σετ δεδομένων μέσω μιας εύχρηστης διεπαφής θα διευκολύνει την ανανέωση των μετα-μοντέλων και σε συνδυασμό με την προηγούμενη λειτουργικότητα θα διευκολύνει τη βελτίωση του συστήματος.
	\item \textbf{Διεπαφή για ενσωμάτωση ευριστικών κανόνων} Οι ευριστικοί κανόνες αποτελούν σημαντική πηγή λήψης αποφάσεων. Μέσω αυτών ενσωματώνεται στο πείραμα η εμπειρία της βιβλιογραφίας. Είναι λοιπόν επιθυμητό οι ευριστικοί κανόνες του συστήματος να ανανεώνονται τόσο με βάση τη βιβλιογραφία όσο και με τις επιθυμίες του χρήστη, λειτουργικότητα που θα οδηγήσει σε ένα πιο ρυθμιζόμενο σύστημα. Σημαντική είναι η γλώσσα στην οποία θα συντάσσονται οι ευριστικοί κανόνες και ο τρόπος με τον οποίο θα αποθηκεύονται σε μία βάση. 
\end{itemize}

Τέλος, ενδιαφέρον παρουσιάζει η δυνατότητα αυτόματης παραγωγής ευριστικών κανόνων. Αναγνωρίζουμε πως οι ευριστικοί κανόνες αποτελούν μετα-γνώση, η οποία δεν έχει μοντέλο παραγωγής, αλλά διατυπώνεται σε μορφή ποσοτικών κανόνων που προκύπτουν από πειραματική εμπειρία. Καθώς διαθέτουμε ένα σύστημα εκτέλεσης πειραμάτων αναγνωρίζουμε τη δυνατότητα ανατροφοδότησης του συστήματος με την απόδοση των επιλογών του. Όπως λοιπόν η κοινότητα των αναλυτών δεδομένων πειραματίζεται, παρατηρεί και συμπεραίνει για τη δημιουργία ευριστικών κανόνων, έτσι και το σύστημά μας, με χρήση Αναγνώρισης Προτύπων, μπορεί να συλλάβει τους δικούς του ευριστικούς κανόνες εκ των οποίων θα επωφελείται το ίδιο για τη λήψη αποφάσεων. 
