\chapter{Εξαγωγή διαστημάτων πρόβλεψης από μοντέλα παλινδρόμησης}
\label{appendix:Intervals}

Το διάστημα πρόβλεψης αποτελεί μια εκτίμηση για το διάστημα στο οποίο θα βρεθούν μελλοντικές παρατηρήσεις ενός πληθυσμού με μία συγκεκριμένη πιθανότητα.

Αν θεωρήσουμε μία κανονική κατανομή  $\mathcal{N} (\mu, \sigma)$, τότε το διάστημα πρόβλεψης για πιθανότητα $\gamma$ προκύπτει με τη βοήθεια της τυπικής κανονικής κατανομής Z, για την οποία τα τεταρτημόρια είναι προ-υπολογισμένα.
\begin{equation}
\gamma = P\big(l < X < u\big) = P\Bigg(\frac{l-\mu}{\sigma} < \frac{X-\mu}{\sigma} < \frac{u-\mu}{\sigma}\Bigg)= P\Bigg(\frac{l-\mu}{\sigma} < Z < \frac{u-\mu}{\sigma}\Bigg)
\end{equation}

Επομένως
\begin{equation}
\frac{l-\mu}{\sigma} = -z \qquad\text{και}\qquad \frac{u-\mu}{\sigma} = -z
\end{equation}

και το διάστημα πρόβλεψης ορίζεται ως:
\begin{equation}
	[\mu - z \sigma, \mu + z \sigma ]
\end{equation}

\paragraph{Από γραμμικό μοντέλο}
Κατά την εκπαίδευση ενός γραμμικού μοντέλου συνίσταται η επίδειξη κανονικότητας των residuals του μοντέλου, προκειμένου να είναι δυνατός ο υπολογισμός των διαστημάτων πρόβλεψης μέσω της κανονικής κατανομής.

Αν θεωρήσουμε ότι έχουμε $n$ παραδείγματα και $s_y$ είναι η τυπική απόκλιση των residuals του μοντέλου, η οποία ορίζεται ως:
\begin{equation}
\sqrt{\frac{\sum(y_i - \hat{y_i})^2}{n-2}}
\end{equation}
τότε το διάστημα πρόβλεψης δίνεται από τον τύπο:
\begin{equation}
\hat{y}\pm t^*_{n-2} s_y \sqrt{1 + \frac{1}{n} + \frac{(x^* - \bar{x})^2}{(n-1) s_x^2}}
\end{equation}
όπου ο όρος $t$ αντιστοιχεί στο δείκτη t value, το t-statistic της μηδενικής υπόθεσης ο συντελεστής του μοντέλου παλινδρόμησης να είναι μηδενικός.   

[ΒΑΛΕ ΚΑΙ ΑΛΛΑ ΣΤΑΤΙΣΤΙΚΣ ΚΑΙ ΟΝΟΜΑΣΕ ΑΥΤΟ ΠΟΥ ΧΡΗΣΙΜΟΠΟΙΗΣΕΣ]

\paragraph{Από SVM} Ένα μοντέλο παλινδρόμησης παραγόμενο με SVM διαφέρει από ένα συμβατικό γραμμικό μοντέλο ως προς τον τρόπο διαχείρισης των δεδομένων, τα οποία μετασχηματίζει σε ένα νέο χώρο σύμφωνα με τη συνάρτηση πυρήνα. Ως αποτέλεσμα οι τεχνικές εξαγωγής διαστημάτων πρόβλεψης που περιγράψαμε δεν είναι εφαρμόσιμες.

Η βιβλιογραφία περιέχει διάφορες προσπάθειες απόδοσης πιθανοτικής εξόδου στον αλγόριθμο SVM, όπως ο αλγόριθμος του \citet{Platt99probabilisticoutputs} για μοντέλα ταξινόμησης και η μέθοδος των \citet{Jiang:2008:ECI:1390681.1390698} και \citet{Lin04simpleprobabilistic} για παλινδρόμησης. Εμείς βασίσαμε τα πειράματά μας στη δεύτερη μέθοδο, καθώς εφαρμόζεται από την επικρατέστερη βιβλιοθήκη εκπαίδευσης SVM μοντέλων, τη LIBSVM~\footnote{https://www.csie.ntu.edu.tw/~cjlin/libsvm/}, η οποία επίσης αποτελεί τη βάση της βιβλιοθήκης kernlab που χρησιμοποιήσαμε.

Η μέθοδος αυτή μοντελοποιεί τα σφάλματα των προβλέψεων (residuals) ως
\begin{equation}
\zeta = y - \hat{f(x)}
\end{equation}

όπου y η κλάση μίας παρατήρησης και $\hat{f}(x)$ η πρόβλεψη για αυτήν. Στόχος της ανάλυσης είναι η εύρεση της κατανομής της τυχαίας μεταβλητής $\zeta$, ώστε η κατανομή του $y$ να προκύψει από τη συνέλιξη των επιμέρους κατανομών. 

Όπως περιγράφουν οι \citet{Chang:2011:LLS:1961189.1961199} ,ο υπολογισμός της κατανομής γίνεται παράγοντας τα σφάλματα εκτός δείγματος (out-of-sample residuals) με τη χρήση cross-validation και αναγνωρίζοντας την κατανομή που τα περιγράφει. Σύμφωνα με τα πειράματα των \citet{Lin04simpleprobabilistic} καταλληλότερη κατανομή είναι η λαπλασιανή, η οποία για τυχαία μεταβλητή $z$ περιγράφεται από τον τύπο
\begin{equation}
	p(z) = \frac{1}{2\sigma} e^{-\frac{\abs{z}}{\sigma}}
	\label{eq:laplace}
\end{equation} 
όπου $\sigma$ η παράμετρος κλιμάκωσης (scale parameter), η τιμή της οποία δίνεται από τη τεχνική της μέγιστης πιθανοφάνειας ως:
\begin{equation}
\sigma = \frac{\sum_{i=1}^{l} \abs{\zeta_{i}}}{l}
\end{equation}
όπου $l$ το πλήθος των residuals.

Για να υπολογίσουμε το διάστημα πρόβλεψης με βεβαιότητα $1-2\sigma$ θα χρειαστεί να προσδιορίσουμε το $\sigma$-μόριο της κατανομής, $p_s$, το οποίο στη γενική περίπτωση μιας συμμετρικής μεταβλητής Z δίνεται από τον τύπο
\begin{equation}
\int_{-\infty}^{p_z}p(z)dz = 1-s
\end{equation}
που με χρήση της σχέσης \ref{eq:laplace} δίνει το διάστημα
\begin{equation}
	(\sigma \ln{2 s}, -\sigma \ln{2 s})
\end{equation}
